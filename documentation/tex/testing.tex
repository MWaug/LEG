This section describes the basic procedure for generating and running tests. 
It covers running a provided test, useful GDB commands, and compiling handwritten tests.

% TODO: Fill out these sections
\subsection{Relevant Files}
\begin{itemize}
\item \textbf{debug\_lockstep/debug.sh} The entry point for the debugging framework. 
\item \textbf{debug\_lockstep/unitTests/*} Randomized tests for every instruction. 
Each test is named with the instruction or instruction type followed by the number of instructions in the test. 
Each test has four or five files
\begin{itemize}
\item \texttt{test.s}, the randomly generated assembly.
\item\texttt{test.dat}, the corresponding hexadecimal machine code.
\item \texttt{test.dump}, the disassembled machine code.
\item \texttt{test.elf} (optional), which contains information about how interrupts are raised in exception handler tests.
\item \texttt{test.bin}, the binary file that runs on the processor and QEMU.
\end{itemize}

\item \textbf{debug\_lockstep/customTests/compileTest.sh} Script for compiling tests for LEG from assembly source. See Section \ref{sec:maketest}.

\item \textbf{debug\_lockstep/customTests/makeRandomAssembly.py} Script to generate random assembly instructions to test on LEG. See Section \ref{sec:randasm}.
\end{itemize}

\subsection{Basic Operation on Provided Tests}
This section demonstrates the most basic operation of the testing framework.
It will allow you to run a test from the terminal and debug in gdb at a register and instruction level.
\begin{enumerate}
\item From your local copy of the LEG repository, enter the \texttt{debug\_lockstep} directory.
\item Provided tests are stored in the \texttt{unitTests} directory. 
 To run a provided test, type 
 \begin{verbatim}./debug.sh -t unitTests/testname.bin\end{verbatim}
 at the terminal. For example, to run the test consisting of 1000 random \texttt{ADC} instructions, enter 
 \begin{verbatim}./debug.sh -t unitTests/adc_1000.bin\end{verbatim}
\item LEG will now be compiled by ModelSim and instances of QEMU and GDB will be started.
 If there are no compilation errors, you will be at the prompt of gdb modified with LEG-specific debugging commands.
 Many of these are explained further in Section \ref{sec:leggdb}
 
 The most basic command is \texttt{leg-lockstep}, which runs LEG until a bug is found.
 Enter this now.
\item The test will print a `\texttt{.}' after each successfully executed instruction.
 If the program completes successfully, a \texttt{PASSED} message will appear.
 Otherwise debugging information will be printed, including the last correct register state, the current incorrect LEG register state, and the expected register state.
\item Debug with any GDB command, or type \texttt{leg-stop} to exit.
\end{enumerate}

This guide cannot present a complete tutorial on using GDB, but a few examples of the most useful commands are listed in Table \ref{table:gdb} below.

\begin{table}
\centering
\begin{tabular}{|l|p{10cm}|}
\hline \textbf{command} & \textbf{effect} \\ 
\hline \texttt{stepi} & Execute one instruction \\ 
\hline \texttt{i r} & Print register contents of the current mode, including cpsr \\ 
\hline \texttt{x/10i \$pc} & Print the next 10 instructions \\ 
\hline \texttt{x/10w \$sp-20} & Print 10 words of memory, starting 20 addresses below the stack pointer \\ 
\hline \texttt{x/5b 0xfffdc} & Print 5 bytes of memory, starting at \texttt{0xfffdc} \\ 
\hline \texttt{b *0x220} & Set a breakpoint at address \texttt{0x220} \\ 
\hline 
\end{tabular} 
\caption{Examples of common GDB commands}
\label{table:gdb}
\end{table}

\subsection{Creating Tests}
Custom tests can be created from Assembly source and run on LEG using the tools in \texttt{debug\_lockstep/customTests}.

\subsubsection{Generating Assembly}\label{sec:randasm}
The script \texttt{makeRandomAssembly.py} can be used to easily create random instructions to test LEG with a vast combination of cases.
Because interrupt handlers, a small stack, and all processor modes are automatically initialized, the output of the script can also be extended with your own additional test vectors.
For example, to create 1234 random add and load instructions and output the program to \texttt{add\_load.s}, run
\begin{verbatim}python makeRandomAssembly.py -i add ldr -n 1234 > add_load.s\end{verbatim}
Run the script with \texttt{-h} to print full usage info.

NOTE: Sometimes immediates produced by the script are not recognized by the compiler. 
In these cases you must re-run \texttt{makeRandomAssembly.py} or manually change the immediate value that fails compilation.

\subsubsection{Compiling Assembly}\label{sec:maketest}
Any assembly code compatible with ARMv5 excluding Thumb can tested on LEG.
Compiling the code uses the ARM bare metal development toolchain that was previously installed in Section \ref{sec:install}.
Simply run \texttt{compileTest.sh} in \texttt{debug\_lockstep/customTests} with the basename of the assembly code you want to compile. Note that the \texttt{.s} extension is not included.

\begin{verbatim}./compileTest.sh add_load\end{verbatim}

The compiled files will be placed in the \texttt{tests} directory. 
The test can now be run on LEG from \texttt{debug\_lockstep}:

\begin{verbatim}./debug.sh -t customTests/tests/add_load.bin\end{verbatim}

\subsection{Debugging Tools}

