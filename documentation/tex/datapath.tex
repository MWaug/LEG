\label{sec:dp}
This section gives an overview of the LEG datapath. 
The datapath is pipelined with five stages: Fetch, Decode, Execute, Memory, and Writeback. 
The stages can be flushed or stalled by the Hazard unit, and data flow is selected by the Controller module.

\noindent Datapath components are illustrated in black on the processor diagram \\\texttt{documentation/diagrams/pipelinedfull.pdf}

\subsection{Relevant Files}

\begin{tabular}{|l|p{120mm}|}
\hline \textbf{File}  & \textbf{Description} \\ 
\hline datapath.sv & Contains all datapath logic and signals for the LEG processor.  \\ 
\hline adder.sv & A parameterized adder. \\ 
\hline addressdecode.sv & Decodes a 4 bit register number and the current processor mode to a one-hot register file select signal. \\ 
\hline regfile.sv & A 32 entry by 32 bit register file containing the 31 general purpose registers and one scratch register for micro operations. Located in the Decode stage. \\ 
\hline extend.sv & Handles extending immediate values of varying width to 32 bits. Located in the Decode stage. \\ 
\hline rotator.sv & A funnel shifter that rotates the 32 bit immediate in data processing instructions. Located in the Decode stage. \\ 
\hline barrel\_shifter.sv & An efficient two stage logarithmic barrel shifter that implements LSR, ASR, LSL, ROR, and RRX shift types. Located in the Execute stage. \\ 
\hline alu.sv & Arithmetic Logic Unit that computes and selects between addition, AND, OR, and XOR operations. Located in the Execute stage. \\ 
\hline zero\_counter.sv & Outputs the number of leading zeros in its input. This module is a good target for optimization in class projects. Located in Execute stage. \\ 
\hline data\_replicator.sv & Selects the size of data required for a memory operation and replicates it to fill the available byte positions. Located in Execute stage. \\ 
\hline data\_selector.sv & Masks data words read from memory and extends them to 32 bits. Located in Writeback stage. \\ 
\hline 
\end{tabular} 

\subsection{Fetch Stage}
The Fetch stage reads an instruction from instruction memory using the register 15, the program counter (PC).
This PC is selected from several sources based on branch or exception status.
The instruction memory is shown in the datapath in the processor diagram, but is actually implemented externally to the processor core. 
See Section \ref{sec:i-cache} for more information.

\subsection{Decode Stage}
In the Decode stage registers are read and immediates are extended and rotated to the form required for arithmetic processing. 
Note that the instruction processed in the decode stage may not match the instruction read in the execute stage due to micro operation decoding (see Section \ref{sec:uop}).

\subsection{Execute Stage}
The Execute stage processes instruction operands into a resultant value. 
First forwarded register values are optionally selected based on signals from the Hazard unit (Section \ref{sec:fwd}).
Shifts or rotates are applied to the second operand and results are computed by the zero counter and ALU. 
The second operand is also processed as data for a potential memory operation. 
Finally, the output is selected from the correct functional unit based on the instruction type.

\subsection{Memory Stage}
The values computed in the Execute stage may be used as data and address in the Memory stage.
The result may also bypass the data memory when a store operation is not performed.
As with the instruction memory, the data memory is drawn in the datapath but is actually a separate functional unit described in Section \ref{sec:d-cache}.

\subsection{Writeback Stage}
In the Writeback stage the result of the Memory stage is fed back to be written to registers or the PC.
